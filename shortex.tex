\documentclass{article}

%%%%%%%%%%%%%%%%%%%%%%%%%%%%%%%%%%%%%%%%%%%%%%
%%  shortex.tex
%%  Latex packages and macros for
%%  more efficient, readable math
%%
%%  Copyright (c) 2023 Trevor Campbell, 
%%  Jonathan Huggins, Jeffrey Negrea
%%%%%%%%%%%%%%%%%%%%%%%%%%%%%%%%%%%%%%%%%%%%%%

\usepackage[commenters={A,B,C,D,E,F,G,H,I,J}]{shortex}

\title{The \texttt{ShorTeX} package}
\author{Trevor Campbell, Jonathan Huggins, and Jeff Negrea}
\date{Updated \today}


\begin{document}

\maketitle

\babs

The purpose of the ShorTeX (meta)package is to make the process of typesetting
typical mathematical documents in \LaTeX~more efficient, and the resulting
code easier to read.  It achieves this by 
(1) providing an
extensive, internally consistent, and easy to learn set of macro
shorthands and custom commands, and 
(2) incorporating a set of packages that are
dedicated to reducing manual coding effort.
\eabs


\tableofcontents

\section{Usage and package options}\label{sec:usage}

Put a copy of \texttt{shortex.sty} in the folder
alongside your other document files, and
include ShorTeX in the document by adding \verb!\usepackage{shortex}! to the preamble.
\textbf{Do not install ShorTeX system-wide;} this package has not yet reached a stable version 1.0,
and we are updating things regularly without any guarantee of backwards
compatibility. 
\textbf{You must compile your document 4 times when using ShorTeX} to ensure that equation
numbers and references update properly.


ShorTeX will include and configure many common packages for you (e.g., \texttt{graphicx}, \texttt{subcaption}, \texttt{hyperref},
\texttt{algorithm}, \texttt{algpseudocode}, \texttt{amsmath}, among others),
so you do not need to explicitly include and set these up yourself.
If you are writing a document that must use a specific style file (e.g., for a conference or journal) that itself
includes some of these packages, we recommend editing those style files to remove the package imports.

The ShorTeX package has a few options:
\bdesc
\item[\texttt{noautonum}] Do not include \texttt{autonum.sty}. This disables automatic equation numbering.
\item[\texttt{blacklinks}] Switch hyperlinks, citations, references, etc.~to be typeset in black font. The default is dark blue.
\item[\texttt{draft}] Turn on draft mode for \texttt{graphicx}, \texttt{hyperref} (with placeholder figures etc).
\item[\texttt{nomathfontdefaults}] Disable default font style/accent combinations (see \cref{sec:fontstylesaccents} for details).
\item[\texttt{commenters}] Specify names of commenters for comment commands (see \cref{sec:commenting} for details).
\item[\texttt{hidecomments}] Suppress comments (see \cref{sec:commenting} for details).
\edesc

\newpage
\section{Environments}

\LaTeX~documents often include a lot of verbose code
related to creating environments (\verb!\begin{blah}...\end{blah}!). ShorTeX provides a set of 
shortened macros for common environments.
Each shortened begin and end command starts with \verb!\b...! and \verb!\e...!, respectively.
Note that all theorem-like environments (theorem, lemma, proposition, etc.) 
are numbered by default; unnumbered versions can be obtained by appending a \verb!u!. For example,
\verb!\bthmu...\ethmu! creates an unnumbered theorem environment, while
\verb!\blemu...\elemu! creates an unnumbered lemma environment.

{\small
\bcent
\btabr{@{}ll@{}}
\toprule
Environment & Syntax \\ \midrule
abstract & \verb!\babs...\eabs!\\ \midrule
itemize & \verb!\bitem...\eitem!\\
enumerate & \verb!\benum...\eenum!\\
description & \verb!\bdesc...\edesc!\\ \midrule
algorithm & \verb!\balg...\ealg!\\
algorithmic & \verb!\balgc...\ealgc!\\ \midrule
table & \verb!\btab...\etab!\\
subtable & \verb!\bsubtab...\esubtab!\\
tabular & \verb!\btabr...\etabr!\\ \midrule
figure & \verb!\bfig...\efig!\\
figure* & \verb!\bfigs...\efigs!\\
subfigure & \verb!\bsubfig...\esubfig!\\ \midrule
center & \verb!\bcent...\ecent!\\ \midrule
align & \verb!\[...\]!\\ 
inline math & \verb!$...$!\\ \midrule
\multicolumn{2}{c}{\emph{Note: These are numbered theorem-like environments.}}\\
\multicolumn{2}{c}{\emph{For unnumbered, append a \texttt{u}: e.g.,} \texttt{bthmu...ethmu}.}\\
theorem & \verb!\bthm...\ethm!\\ 
lemma & \verb!\blem...\elem!\\
proposition & \verb!\bprop...\eprop!\\
corollary & \verb!\bcor...\ecor!\\
conjecture & \verb!\bconj...\econj!\\
definition & \verb!\bdefn...\edefn!\\
assumption & \verb!\bassum...\eassum!\\
example & \verb!\bexa...\eexa!\\
remark & \verb!\brmk...\ermk!\\
fact & \verb!\bfact...\efact!\\
exercise & \verb!\bexer...\eexer!\\ \midrule
proof & \verb!\bprf...\eprf!\\
proofof & \verb!\bprfof{\cref{theorem_label}}...\eprfof!\\  \midrule
matrix & \verb!\bmat...\emat!\\
bmatrix & \verb!\bbmat...\ebmat!\\
pmatrix & \verb!\bpmat...\epmat!\\
\bottomrule
\etabr
\ecent
}


\newpage
\section{Math font styles and accents}\label{sec:fontstylesaccents}

Applying accents (e.g., hats $\s[h]a$, tildes $\s[t]a$, bars $\s[b]a$)
and changing fonts (e.g., doublestroke $\s[d]A$, caligraphic $\s[c]A$, and bold $\s[k]A$)
is quite cumbersome in standard \LaTeX. For example, the code to make a tilde caligraphic A,
$\widetilde{\mathcal{A}}$
is \verb!\widetilde{\mathcal{A}}!. By itself that code is not too bad, but many such characters 
in a large mathematical expression results in unreadable code.

\subsection{Default font, accent combinations}

ShorTeX defines an efficient syntax for changing math fonts and applying accents to characters.
The syntax takes the form \verb!\s<modifiers>character!, where \verb!<modifiers>! is a set of single characters
that represent font/accent modifications to \verb!character!. 
For example, the code for tilde caligraphic A is \verb!\stcA! where \verb!t! represents ``tilde,'' \verb!c! represents
``caligraphic,'' and \verb!A! is the character to typeset. The table below shows the modifier characters that
are implemented by default with all upper/lowercase Greek and Roman characters.
Because \verb!b! is already used for ``bar,'' ShorTeX uses \verb!k! for ``bold'' (arising from the use of \verb!k! to represent
``black'' in plotting packages across many languages, which itself arises from the ``K'' in CMYK).

\bcent
\btabr{@{}llll@{}}
\toprule
Style & Modifier & Example & Typeset Example \\ \midrule
caligraphic (\verb!mathcal!) & \verb!c! & \verb!\scA! & $\scA$ \\
bold (\verb!mathbf!) & \verb!k! & \verb!\skA! & $\skA$\\
doublestroke (\verb!mathbb!) & \verb!d! & \verb!\sdA! & $\sdA$\\ \midrule
Accent & Modifier & Example & Typeset Example \\ \midrule
hat (\verb!widehat!) & \verb!h! & \verb!\shA! & $\shA$\\
tilde (\verb!widetilde!) & \verb!t! & \verb!\stA! & $\stA$\\
bar (\verb!widebar!) & \verb!b! & \verb!\sbA! & $\sbA$\\
dot (\verb!dot!) & \verb!o! & \verb!\soA! & $\soA$\\
\bottomrule
\etabr
\ecent

In addition, ShorTeX allows the use of combinations of any one of the above fonts with any one of the above accents.
For example, we can combine the caligraphic style \verb!c! with all the accents
via \verb!\stcA!, \verb!\socA!, \verb!\sbcA!, \verb!\shcA!:
\[
\stcA \quad \socA \quad \sbcA \quad \shcA.
\]
To disable these default shortcuts, pass the \texttt{nomathfontdefaults} option to ShorTeX.

\subsection{Advanced usage: flexible font, accent combinations}

For most users, the default font style/accent combinations that come with ShorTeX will suffice.
However, the \verb!\s...! commands are actually implemented under the hood using
a more flexible \verb!\s[modifiers]character! command that can take a wide variety of
combinations of font/accent modifiers. For example, the code for tilde 
caligraphic A is \verb!\s[tc]A! where \verb!t! represents ``tilde,'' \verb!c! represents
``caligraphic,'' and \verb!A! is the character to typeset.

\bcent
\btabr{@{}llll@{}}
\toprule
Style & Modifier & Example & Typeset Example \\ \midrule
caligraphic (\verb!mathcal!) & \verb!c! & \verb!\s[c]A! & $\s[c]A$ \\
bold (\verb!mathbf!) & \verb!k! & \verb!\s[k]A! & $\s[k]A$\\
doublestroke (\verb!mathbb!) & \verb!d! & \verb!\s[d]A! & $\s[d]A$\\
fraktur (\verb!mathfrak!) & \verb!f! & \verb!\s[f]A! & $\s[f]A$\\ \midrule
Accent & Modifier & Example & Typeset Example \\ \midrule
hat (\verb!widehat!) & \verb!h! & \verb!\s[h]A! & $\s[h]A$\\
tilde (\verb!widetilde!) & \verb!t! & \verb!\s[t]A! & $\s[t]A$\\
bar (\verb!widebar!) & \verb!b! & \verb!\s[b]A! & $\s[b]A$\\
dot (\verb!dot!) & \verb!o! & \verb!\s[o]A! & $\s[o]A$\\
\bottomrule
\etabr
\ecent

These style modifiers can be combined; the underlying code is flexible enough that
it will happily produce a wide variety of combinations, including those that aren't very sensible.

\emph{Note: modifiers are applied in the reverse of the order in which they appear; 
the modifier furthest to the right is applied first. This matches the order that 
the corresponding commands would appear in TeX code.}

\bcent
\btabr{@{}llll@{}}
\toprule
Style/Accent & Modifier & Example & Typeset Example \\ \midrule
caligraphic tilde & \verb!ct! & \verb!\s[ct]A! & $\s[ct]A$ \\
bold hat & \verb!kh! & \verb!\s[kh]A! & $\s[kh]A$\\
tilde hat  & \verb!ht! & \verb!\s[ht]A! & $\s[ht]A$\\
hat tilde   & \verb!th! & \verb!\s[th]A! & $\s[th]A$\\
fraktur dot   & \verb!of! & \verb!\s[of]A! & $\s[of]A$\\
\bottomrule
\etabr
\ecent

The default \verb!\s...! commands included in ShorTeX are compiled
in advance from the more general \verb!\s[...]...!
via the \verb!\parsefontstylestrings! command; this lets us avoid having
to type \texttt{[]} for common style/accent combinations.
For example, if we use ``bold hat'' symbols frequently,
we might want to use commands like
\verb!\skh...!  instead of \verb!\s[kh]...!.
We can accomplish this using the \verb!\parsefontstylestrings! command,
with syntax
\begin{verbatim}
\parsefontstylestrings{{<fstr1>}{<fstr2}...}{<alphabet>}
\end{verbatim}
For example, to define ``bold hat'' and ``caligraphic hat'' styles
for the characters A, B, C, and D, we would use the command 
\begin{verbatim}
\parsefontstylestrings{{kh}{ch}}{ABCD}
\end{verbatim}
\parsefontstylestrings{{kh}{ch}}{ABCD}
and then in the \LaTeX~document, use the commands
\verb!\skhA \skhB \skhC \skhD! and
\verb!\schA \schB \schC \schD! 
to obtain the following symbols:
\[
\skhA \skhB \skhC \skhD 
\schA \schB \schC \schD 
\]
As another example, for ``bold hat'' applied to $\alpha$, $\beta$, and $\gamma$, we would use the syntax
\begin{verbatim}
\parsefontstylestrings{{kh}}{{\alpha}{\beta}{\gamma}}
\end{verbatim}
\parsefontstylestrings{{kh}}{{\alpha}{\beta}{\gamma}}
and then in the \LaTeX~document, use the commands
\verb!\skhalpha \skhbeta \skhgamma!
to obtain the following symbols:
\[
	\skhalpha \skhbeta \skhgamma
\]

For convenience we also provide a few common alphabets of symbols 
for use in the \verb!\parsefontstylestrings! command.
Note that not every Greek character has a lowercase or uppercase version (in cases where it is
identical to its Roman counterpart). Also note that we use ShorTeX Greek letter syntax;
see \cref{sec:greeks}.

\bcent
\btabr{@{}lll@{}}
\toprule
Syntax & Characters  \\ \midrule
\verb!\lowercaseRoman! & abcdefghijklmnopqrstuvwxyz \\
\verb!\uppercaseRoman! & ABCDEFGHIJKLMNOPQRSTUVWXYZ \\
\verb!\lowercaseGreek! & alpha,beta,gamma,delta,eps,zeta,\\
& eta,theta,iota,kappa,lam,mu,nu,xi,pi,rho,\\
& sigma,tau,ups,phi,chi,psi,omega,veps,vtheta,\\
& vkappa,vpi,vrho,vsigma,vphi\\
\verb!\uppercaseGreek! & Gamma,Delta,Theta,Lam,Xi,\\
&Pi,Sigma,Ups,Phi,Psi,Omega\\
\bottomrule
\etabr
\ecent

\newpage
\section{Commenting}\label{sec:commenting}
ShorTeX defines four types of document markup that can be used: 
\emph{comment}, \emph{emphasized comment}, \emph{margin comment}, and \emph{highlight}.
This is a lightweight alternative to some more common todo packages (e.g., \texttt{todonotes}).
In order to create comments, you must pass the \verb!commenters! option to the package, and specify
an identifier for each commenter. For example, one could specify three commenters 
(named A, B, C) using the command\\
\\
\verb!\usepackage[commenters={A,B,C}]{shortex}!
\\\\
For each commenter, there are four commands (one for each markup type). The table
below contains examples for commenter ``A''. Notice that each comment is tagged with a number
(specific to each commenter) for easy referencing.

\bcent
\btabr{@{}llll@{}}
\toprule
Comment Type & Syntax & Example & Typeset Example\\ \midrule
comment & \verb!\c...{comment}! & \verb~\cA{hello!}~ & \cA{hello!} \\ 
emphasized comment & \verb!\e...{comment}! & \verb~\eA{hello!}~ & \eA{hello!} \\ 
margin comment & \verb!\m...{comment}! & text\verb~\mA{hello!}~ & text\mA{hello!} \\ 
highlight & \verb!\h...{text}! & \verb~\hA{hello!}~ & \hA{hello!} \\ 
\bottomrule
\etabr
\ecent

Note that each commenter gets their own color. Currently ShorTeX supports nine different commenter colors,
and will wrap around back to the first color if the number of commenters exceeds nine:
\bitem
\setlength{\itemsep}{0pt}
\item \cA{example}\eA{emphasized example}more text\mA{margin example}
\item \cB{example}\eB{emphasized example}more text\mB{margin example}
\item \cC{example}\eC{emphasized example}more text\mC{margin example}
\item \cD{example}\eD{emphasized example}more text\mD{margin example}
\item \cE{example}\eE{emphasized example}more text\mE{margin example}
\item \cF{example}\eF{emphasized example}more text\mF{margin example}
\item \cG{example}\eG{emphasized example}more text\mG{margin example}
\item \cH{example}\eH{emphasized example}more text\mH{margin example}
\item \cI{example}\eI{emphasized example}more text\mI{margin example}
\item Back to first color: \cJ{example}\eJ{emphasized example}more text\mJ{margin example}
\eitem
You can also disable all comments to see a clean version of the current 
document using the \verb!suppresscomments! package option.
This option will blank out all comments and render highlighted text normally.

\newpage
\section{Referencing figures, equations, tables, etc.}

ShorTeX includes the \texttt{cleveref} package, which simplifies the process
of typesetting references. Use the \verb!\cref! command (or \verb!\Cref! at the beginning of a sentence) 
to automatically typeset the names of the objects you reference (including properly handling multiple references). 
For example, if \verb!\label{fig:first}! is applied to the first figure in the document,
\begin{verbatim}
In \cref{fig:first}, you can see...
\end{verbatim}
would typeset as ``In Fig.~1, you can see...''
Similarly, if \verb!\label{thm:first}! references a theorem and \verb!\label{second_result}! references
a lemma, 
\begin{verbatim}
\cref{thm:first,lem:second} show that...
\end{verbatim}
will typeset as ``Theorem 1 and Lemma 2 show that...''
This works for many different reference types (Figure, Algorithm, Equation, Table, etc),
and can be extended if needed. See the \texttt{cleveref} documentation 
at \url{https://ctan.org/pkg/cleveref?lang=en} and the homepage at \url{https://www.dr-qubit.org/cleveref.html} 
for more information.

ShorTeX also includes the \texttt{autonum} package, which simplifies the process of 
equation numbering. Typically when you typeset equations, you have to choose between 
\verb!$...$!, \verb!$$...$$!, 
\verb!\begin{align}...\end{align}!, 
\verb!\begin{aligned}...\end{aligned}!, 
\verb!\begin{equation}...\end{equation}!, 
not to mention starred versions of those environments 
and \verb!\nonumber!/\verb!\notag! commands, depending 
on whether/where you want equation numbers,

The \texttt{autonum} package automatically decides which equations to provide
numbers based on \textit{which equations you reference}. So when using ShorTeX,
you only need two commands for math mode: single dollar signs \verb!$...$! for
inline math, and \texttt{align} environments (redefined in ShorTeX to be
\verb!\[...\]!) for display math.\footnote{There are
differences between how \texttt{align} and 
other math display environments typeset equations. I have not ever
encountered a case where it mattered much. If you are very picky about typesetting,
note that ShorTeX does not \emph{disable} any functionality, so you 
can use other environments anywhere you feel it is necessary.}

For example, if you create 
the following display math,
\begin{verbatim}
\[
   a+b = c \label{eq:the_equation}
\]
\end{verbatim}
then if you use the command \verb!\cref{eq:the_equation}! somewhere
in the document, that equation will automatically be assigned a number. If not, it
won't get a number. See the \texttt{autonum} package 
documentation \url{https://ctan.org/pkg/autonum?lang=en} for more information.



\section{Math mode macros}

\subsection{Delimiters}

Mathematics in \LaTeX~often includes quite a few delimiters (parentheses, brackets, curly brackets, etc.).
A very common usage of these involves the \verb!\left...\right! commands for automatic sizing, as well as \verb!\middle! 
for sizing characters in between delimiters.
One can also use \verb!\bigl...\bigr!, \verb!\Bigl...\Bigr!, \verb!\biggl...\biggr!, \verb!\Biggl...\Biggr! to control sizing manually.
ShorTeX creates shorthands for these.

\bcent
\btabr{@{}llll@{}}
\toprule
Description & Syntax  \\ \midrule
automatic left, right	& \verb!\lt...\rt!\\        
automatic middle	& \verb!\m...!\\        
big 	& \verb!\lb...\rb!\\
Big & \verb!\lB...\rB! \\ 
bigg & \verb!\lbg...\rbg!\\ 
Bigg & \verb!\lBg...\rBg!\\
\bottomrule
\etabr
\ecent

These can be applied to all the usual delimiter characters.
The following tables demonstrate usage for automatically sized delimiters. 

\bcent
\btabr{@{}llll@{}}
\toprule
Description & Example & Text style & Display style \\ \midrule
parentheses	& \verb!\lt(\frac{x}{y}\rt)!        	& $\lt(\frac{x}{y}\rt)$ 		& $\displaystyle\lt(\frac{x}{y}\rt)$ \\[10pt]
curly brackets 	& \verb!\lt\{\frac{x}{y}\rt\}!    	& $\lt\{\frac{x}{y}\rt\}$ 	& $\displaystyle\lt\{\frac{x}{y}\rt\}$ \\[10pt]
square brackets & \verb!\lt[frac{x}{y}\rt]!        	& $\lt[\frac{x}{y}\rt]$ 	& $\displaystyle\lt[\frac{x}{y}\rt]$ \\[10pt]
pipes & \verb!\lt|frac{x}{y}\rt|!        	& $\lt|\frac{x}{y}\rt|$ 	& $\displaystyle\lt|\frac{x}{y}\rt|$ \\[10pt]
double pipes & \verb!\lt\|frac{x}{y}\rt\|!        	& $\lt\|\frac{x}{y}\rt\|$ 	& $\displaystyle\lt\|\frac{x}{y}\rt\|$ \\[10pt]
angle brackets & \verb!\lt<frac{x}{y}\rt>!        	& $\lt<\frac{x}{y}\rt>$ 	& $\displaystyle\lt<\frac{x}{y}\rt>$ \\[10pt]
middle bar 	& \verb!\E\lt[X \m| Y\rt]!    	& $\E\lt[X \m| Y\rt]$ 	& $\displaystyle\E\lt[X \m| Y\rt]$ \\[10pt]
\bottomrule
\etabr
\ecent

Also note that the spacing around the standard \verb!\left...! and \verb!\right...! delimiters is sometimes a bit odd. For example,
the code \verb!\operatorname{A}\left(x\right)! adds too much space between $\operatorname{A}$ and the parentheses:
\[
	\operatorname{A}\left(x\right)
\]
ShorTeX's shorthands \verb!\lt...! and \verb!\rt...! fix this. The code \verb!\operatorname{A}\lt(x\rt)! typesets as
\[
	\operatorname{A}\lt(x\rt)
\]
The code for the spacing fixes was taken from
\url{https://tex.stackexchange.com/questions/2607/spacing-around-left-and-right}).

ShorTeX also provides a set of common paired delimiters:
\bcent
\btabr{@{}llll@{}}
\toprule
Description				& Syntax 					& Text style 				& Display style \\ \midrule
Round brackets	& \verb!\rbra{\frac{x}{y}}!        	& $\rbra{\frac{x}{y}}$ 		& $\displaystyle\rbra{\frac{x}{y}}$ \\[10pt]
Curly brackets 			& \verb!\cbra{\frac{x}{y}}!    	& $\cbra{\frac{x}{y}}$ 	& $\displaystyle\cbra{\frac{x}{y}}$ \\[10pt]
Square brackets 		& \verb!\sbra{\frac{x}{y}}!        	& $\sbra{\frac{x}{y}}$ 	& $\displaystyle\sbra{\frac{x}{y}}$ \\[10pt]
Norm					& \verb!\norm{\frac{x}{y}}!       	& $\norm{\frac{x}{y}}$ & $\displaystyle\norm{\frac{x}{y}}$	\\[10pt]
Inner product			& \verb!\inner{\frac{x}{y},\frac{y}{z}}!       	& $\inner{\frac{x}{y},\frac{y}{z}}$ & $\displaystyle\inner{\frac{x}{y},\frac{y}{z}}$	\\[10pt]
Absolute value 			& \verb!\abs{\frac{x}{y}}!        	& $\abs{\frac{x}{y}}$ 		& $\displaystyle\abs{\frac{x}{y}}$ \\[10pt]
Floor					& \verb!\floor{\frac{x}{y}}!        	& $\floor{\frac{x}{y}}$ 		& $\displaystyle\floor{\frac{x}{y}}$ \\[10pt]
Ceiling 				& \verb!\ceil{\frac{x}{y}}!        	& $\ceil{\frac{x}{y}}$ 		& $\displaystyle\ceil{\frac{x}{y}}$ \\[10pt]
\bottomrule
\etabr
\ecent
These paired delimiters are defined using the \verb!\DeclarePairedDelimiter! command, so 
one can use an asterisk for automatic sizing, or place a size specification in square brackets.
For example, \verb!\rbra{\frac{x}{y}}!, \verb!\rbra*{\frac{x}{y}}!, and \verb!\rbra[\Bigg]{\frac{x}{y}}! typeset as
\[
	\rbra{\frac{x}{y}}, \quad \rbra*{\frac{x}{y}}, \quad \rbra[\Bigg]{\frac{x}{y}}.
\]


\newpage
\subsection{Greek characters and variants}\label{sec:greeks}

ShorTeX defines shorthands for Greek letters with more than 5 characters in their name (\verb!\epsilon!, \verb!\lambda!, \verb!\upsilon!)
as well as variant characters. Variants are obtained by preceding the usual command with \verb!\v...!.

\bcent
\btabr{@{}lllll@{}}
\toprule
Letter & Syntax & Symbol  \\ \midrule
epsilon & \verb!\eps! & $\eps$ \\
lambda & \verb!\lam,\Lam! & $\lam,\Lam$ \\
upsilon & \verb!\ups,\Ups! & $\ups,\Ups$\\
\bottomrule
\etabr
\ecent

\bcent
\btabr{@{}lll@{}}
\toprule
Letter & Variant Syntax & Variant Symbol  \\ \midrule
epsilon & \verb!\veps! & $\veps$ \\
theta & \verb!\vtheta! & $\vtheta$ \\
kappa & \verb!\vkappa! & $\vkappa$ \\
pi & \verb!\vpi! & $\vpi$ \\
rho & \verb!\vrho! & $\vrho$ \\
sigma & \verb!\vsigma! & $\vsigma$ \\
phi & \verb!\vphi! & $\vphi$ \\
\bottomrule
\etabr
\ecent


\newpage
\subsection{Sets and set operations}
\bcent
\btabr{@{}lll@{}}
\toprule
Name & Syntax & Symbol  \\ \midrule
reals	& \verb!\reals! & $\reals$ \\
extended reals	& \verb!\xreals! & $\xreals$ \\
rationals & \verb!\rats! & $\rats$\\
integers	& \verb!\ints! & $\ints$ \\
natural numbers	& \verb!\nats! & $\nats$ \\
complex numbers	& \verb!\comps! & $\comps$ \\
measures & \verb!\msrs! & $\msrs$\\
probability measures & \verb!\pmsrs! & $\pmsrs$\\
intersection & \verb!\inter! & $\inter$\\
union & \verb!\union! & $\union$\\
volume	& \verb!\vol! & $\vol$ \\
diameter	& \verb!\diam! & $\diam$ \\
boundary	& \verb!\bdry! & $\bdry$ \\
closure	& \verb!\cl! & $\cl$ \\
span	& \verb!\spann! & $\spann$ \\
cone	& \verb!\cone! & $\cone$ \\
convex hull	& \verb!\conv! & $\conv$ \\
\bottomrule
\etabr
\ecent

\newpage
\subsection{Linear algebra}

\bcent
\btabr{@{}lll@{}}
\toprule
Name & Syntax & Symbol  \\ \midrule
trace	& \verb!\tr A! & $\tr A$ \\
rank	& \verb!\rank A! & $\rank A$ \\
transpose	& \verb!A\T! & $A\T$ \\
inverse transpose	& \verb!A\nT! & $A\nT$ \\
diagonal	& \verb!\diag A! & $\diag A$ \\
adjoint	& \verb!A\adj! & $A\adj$ \\
spectrum	& \verb!\spec A! & $\spec A$ \\
kronecker product & \verb!A\kron B! & $A\kron B$\\
\bottomrule
\etabr
\ecent

\subsection{Calculus}

\bcent
\renewcommand{\arraystretch}{1.5}
\btabr{@{}lll@{}}
\toprule
Name & Syntax & Symbol  \\ \midrule
differential symbol	& \verb!\d x! & $\d x$ \\
partial differential symbol	& \verb!\pd x! & $\pd x$ \\
gradient symbol	& \verb!\grad f! & $\grad f$ \\
derivative	& \verb!\der{x}{y}! & $\der{x}{y}$ \\
$n^\text{th}$ derivative	& \verb!\der[n]{x}{y}! & $\der[n]{x}{y}$ \\
derivative w.r.t.	& \verb!\der{}{y}! & $\der{}{y}$ \\
partial derivative	& \verb!\pder{x}{y}! & $\pder{x}{y}$ \\
$n^\text{th}$ partial derivative	& \verb!\pder[n]{x}{y}! & $\pder[n]{x}{y}$ \\
partial derivative w.r.t. & \verb!\pder{}{y}! & $\pder{}{y}$ \\
Hessian & \verb!\hes{x}{y}{z}! & $\hes{x}{y}{z}$ \\
Hessian w.r.t. & \verb!\hes{}{y}{z}! & $\hes{}{y}{z}$ \\
\bottomrule
\etabr
\ecent


\newpage
\subsection{General probability and statistics}

\bcent
\btabr{@{}lll@{}}
\toprule
Name & Syntax & Symbol  \\ \midrule
i.i.d.	& \verb!\iid! & \iid \\
almost sure	& \verb!\as! & \as \\
almost everywhere	& \verb!\aev! & \aev \\
convergence almost surely	& \verb!\convas! & $\convas$ \\
convergence in probability	& \verb!\convp! & $\convp$ \\
convergence in distribution	& \verb!\convd! & $\convd$ \\
equality in distribution	& \verb!\eqd! & $\eqd$ \\
equality almost surely	& \verb!\eqas! & $\eqas$ \\
probability	& \verb!\P! & $\P$ \\
expectation	& \verb!\E! & $\E$ \\
variance	& \verb!\Var! & $\Var$ \\
covariance	& \verb!\Cov! & $\Cov$ \\
correlation	& \verb!\Corr! & $\Corr$ \\
support	& \verb!\supp! & $\supp$ \\
distributed as	& \verb!\dist! & $\dist$ \\
distributed \iid	& \verb!\distiid! & $\distiid$ \\
distributed independently	& \verb!\distind! & $\distind$ \\
independent &  \verb!\indep! & $\indep$ \\
Kullback-Leibler divergence & \verb!\KL(q||p)! & $\KL(q||p)$\\
Total variation distance & \verb!\TV(q,p)! & $\TV(q,p)$\\
$\nu$-Wasserstein distance & \verb!\Wass_\nu(q,p)! & $\Wass_\nu(q,p)$\\
Hellinger distance & \verb!\Hell(q,p)! & $\Hell(q,p)$\\
\bottomrule
\etabr
\ecent


\newpage
\subsection{Probability distributions}

\bcent
\btabr{@{}lll@{}}
\toprule
Name & Syntax & Symbol  \\ \midrule
Bernoulli	& \verb!\Bern! & $\Bern$ \\
beta	& \verb!\Beta! & $\Beta$ \\
beta prime	& \verb!\Beta'! & $\Beta'$ \\
binomial	& \verb!\Binom! & $\Binom$ \\
categorical	& \verb!\Cat! & $\Cat$ \\
Cauchy	& \verb!\Cauchy! & $\Cauchy$ \\
chi-squared	& \verb!\ChiSq! & $\ChiSq$ \\
Dirichlet	& \verb!\Dir! & $\Dir$ \\
exponential	& \verb!\Exp! & $\Exp$ \\
gamma	& \verb!\Gam! & $\Gam$ \\
inverse gamma	& \verb!\InvGam! & $\InvGam$ \\
geometric	& \verb!\Geom! & $\Geom$ \\
Gumbel	& \verb!\Gum! & $\Gum$ \\
generalized extreme value	& \verb!\GEV! & $\GEV$ \\
Laplace	& \verb!\Lap! & $\Lap$ \\
multinomial	& \verb!\Multi! & $\Multi$ \\
normal	& \verb!\Norm! & $\Norm$ \\
Poisson	& \verb!\Poiss! & $\Poiss$ \\
student-t	& \verb!\StudentT! & $\StudentT$ \\
uniform	& \verb!\Unif! & $\Unif$ \\
von Mises-Fisher	& \verb!\VMF! & $\VMF$ \\
Wishart	& \verb!\Wish! & $\Wish$ \\
inverse Wishart	& \verb!\InvWish! & $\InvWish$ \\
\midrule
Bernoulli process	& \verb!\BeP! & $\BeP$ \\
beta process	& \verb!\BP! & $\BP$ \\
beta prime process	& \verb!\BPP! & $\BPP$ \\
Dirichlet process	& \verb!\DP! & $\DP$ \\
Chinese restauarant process	& \verb!\CRP! & $\CRP$ \\
completely random measure	& \verb!\CRM! & $\CRM$ \\
normalized completely random measure & \verb!\NCRM! & $\NCRM$ \\
gamma process	& \verb!\GamP! & $\GamP$ \\
normalized gamma process	& \verb!\NGamP! & $\NGamP$ \\
Gaussian process	& \verb!\GP! & $\GP$ \\
Pitman-Yor process	& \verb!\PYP! & $\PYP$ \\
Poisson process	& \verb!\PP! & $\PP$ \\
\bottomrule
\etabr
\ecent

\newpage
\subsection{Text in math}

\bcent
\btabr{@{}lll@{}}
\toprule
Name & Syntax & Symbol  \\ \midrule
s.t.	& \verb!x \st y! & $x\st y$ \\
and	& \verb!x \andd y! & $x \andd y$ \\
or	& \verb!x \orr y! & $x \orr y$ \\
with	& \verb!x \with y! & $x \with y$ \\
where	& \verb!x \where y! & $x \where y$ \\
\bottomrule
\etabr
\ecent

\subsection{Other}

\bcent
\btabr{@{}lll@{}}
\toprule
Name & Syntax & Symbol  \\ \midrule
argmax	& \verb!\argmax_{x\in \reals}! & $\argmax_{x\in\reals}$ \\
argmin	& \verb!\argmin_{x\in \reals}! & $\argmin_{x\in\reals}$ \\
esssup	& \verb!\esssup_{x\in \reals}! & $\esssup_{x\in\reals}$ \\
essinf	& \verb!\essinf_{x\in \reals}! & $\essinf_{x\in\reals}$ \\
indicator	& \verb!\1[x=3]! & $\1[x=3]$ \\
sign	& \verb!\sgn x! & $\sgn x$ \\
%defined as	& \verb!\defas! & $\defas$ \\
%defines	& \verb!\defines! & $\defines$ \\
%half	& \verb!\half! & $\half$ \\
%third	& \verb!\third! & $\third$ \\
%quarter	& \verb!\quarter! & $\quarter$ \\
\bottomrule
\etabr
\ecent

\newpage
\subsection{Shrinking whitespace in math}
The command \verb!\squish{<frac>}! in math mode enables you to shrink whitespace in mathematics,
where \verb!<frac>! represents the fraction of whitespace reduction.
Below, the first line is regularly spaced, the second line has \verb!\squish{0.5}!, and the third has \verb!\squish{0.0}!.
\[
	\sqrt{\frac{1^{2}}{0.111222}(0.111222\times1.111163+0.066987^{2}\times0.111222)-1}&= \sqrt{0.111222}\\
	\squish{0.5}\sqrt{\frac{1^{2}}{0.111222}(0.111222\times1.111163+0.066987^{2}\times0.111222)-1}&= \sqrt{0.111222}\\
	\squish{0.0}\sqrt{\frac{1^{2}}{0.111222}(0.111222\times1.111163+0.066987^{2}\times0.111222)-1}&= \sqrt{0.111222}\\
\]

The code for \verb!\squish! was taken from \url{https://tex.stackexchange.com/questions/467942/how-to-squeeze-a-long-equation}.

\subsection{Wide bar}

ShorTeX provides the \verb!\widebar! command to typeset a wide bar accent on top of a character (similar to the 
usual \verb!\widehat! and \verb!\widetilde! commands). Compare to the usual \verb!\bar! and 
\verb!\overline! commands:
\[
	\text{\texttt{widebar}:}\,\, \widebar{A} \qquad \text{\texttt{overline}:} \,\,\overline{A} \qquad \text{\texttt{bar}:} \,\,\bar{A}
\]
The code for \verb!\widebar! was taken from \url{https://tex.stackexchange.com/questions/16337/can-i-get-a-widebar-without-using-the-mathabx-package}.
Note that the shortened style/accent code \texttt{b} in \cref{sec:fontstylesaccents} encodes \verb!\widebar!, not \verb!\bar!.

\newpage
\section{Text mode macros}

\subsection{Common words and names with accents}

\bcent
\btabr{@{}ll@{}}
\toprule
 Syntax & Symbol  \\ \midrule
\verb!\cadlag! & \cadlag \\
\verb!\Frechet! & \Frechet \\
\verb!\Gronwall! & \Gronwall \\
\verb!\Holder! & \Holder \\
\verb!\Ito! & \Ito \\
\verb!\Levy! & \Levy \\
\verb!\Matern! & \Matern \\
\verb!\Nystrom! & \Nystrom \\
\verb!\Renyi! & \Renyi \\
\verb!\Schatten! & \Schatten \\
\bottomrule
\etabr
\ecent



\subsection{Abbreviations with punctuation}

\bcent
\btabr{@{}lll@{}}
\toprule
Name & Syntax & Symbol  \\ \midrule
e.g.,	& \verb!\eg! & \eg \\
et al.	& \verb!\etal! & \etal \\
i.e.,	& \verb!\ie! & \ie \\
a.k.a.	& \verb!\aka! & \aka \\
\bottomrule
\etabr
\ecent



\newpage
\section{Example Document}

TODO: full shortex example and unit testing of all commands at the end

Note: \verb!\bdR! doesn't work

\end{document}

