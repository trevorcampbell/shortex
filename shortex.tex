\documentclass{article}

%%%%%%%%%%%%%%%%%%%%%%%%%%%%%%%%%%%%%%%%%%%%%%
%%  shortex.tex
%%  Latex packages and macros for
%%  more efficient, readable math
%%
%%  Copyright (c) 2023 Trevor Campbell, 
%%  Jonathan Huggins, Jeffrey Negrea
%%%%%%%%%%%%%%%%%%%%%%%%%%%%%%%%%%%%%%%%%%%%%%



\usepackage[commenters={A,B,C,D,E,F,G,H,I,J}]{shortex}

\title{The \texttt{ShorTeX} package}
\author{Trevor Campbell, Jonathan Huggins, and Jeff Negrea}
\date{Updated \today}


\begin{document}

\maketitle

\babs

The purpose of the ShorTeX (meta)package is to make the process of typesetting
typical mathematical documents in \LaTeX~more efficient, and the resulting
code easier to read.  It achieves this by 
(1) providing an
extensive, internally consistent, and easy to learn set of macro
shorthands and custom commands, and 
(2) incorporating a set of packages that are
dedicated to reducing manual coding effort.
\eabs


\tableofcontents

\section{Usage and package options}\label{sec:usage}

Put a copy of \texttt{shortex.sty} in the folder
alongside your other document files, and
include ShorTeX in the document by adding \verb!\usepackage{shortex}! to the preamble.
\textbf{Do not install ShorTeX system-wide;} this package has not yet reached a stable version 1.0,
and we are updating things regularly without any guarantee of backwards
compatibility. 
\textbf{You must compile your document 4 times when using ShorTeX} to ensure that equation
numbers and references update properly.


ShorTeX will include and configure many common packages for you (e.g., \texttt{graphicx}, \texttt{subcaption}, \texttt{hyperref},
\texttt{algorithm}, \texttt{algpseudocode}, \texttt{amsmath}, among others),
so you do not need to explicitly include and set these up yourself.
If you are writing a document that must use a specific style file (e.g., for a conference or journal) that itself
includes some of these packages, we recommend editing those style files to remove the package imports.

The ShorTeX package has a few options:
\bdesc
\item[\texttt{manualnumbering}] Do not include \texttt{autonum.sty}. This disables automatic equation numbering.
\item[\texttt{blackhypersetup}] Switch hyperlinks, citations, references, etc.~to be typeset in black font. The default is dark blue.
\item[\texttt{draft}] Enable \texttt{graphicx} draft mode (with placeholder figures).
\item[\texttt{minimal}] Disable default font style/accent combinations (see \cref{sec:fontstylesaccents} for details).
\item[\texttt{commenters}] Specify names of commenters for comment commands (see \cref{sec:commenting} for details).
\item[\texttt{suppresscomments}] Suppress comments (see \cref{sec:commenting} for details).
\edesc

\newpage
\section{Environments}

\LaTeX~documents often include a lot of verbose code
related to creating environments (\verb!\begin{blah}...\end{blah}!). ShorTeX provides a set of 
shortened macros for common environments.
Each shortened begin and end command starts with \verb!\b...! and \verb!\e...!, respectively.
Note that all theorem-like environments (theorem, lemma, proposition, etc.) 
are numbered by default; unnumbered versions can be obtained by appending a \verb!u!. For example,
\verb!\bthmu...\ethmu! creates an unnumbered theorem environment, while
\verb!\blemu...\elemu! creates an unnumbered lemma environment.

{\small
\bcent
\btabr{@{}ll@{}}
\toprule
Environment & Syntax \\ \midrule
abstract & \verb!\babs...\eabs!\\ \midrule
itemize & \verb!\bitem...\eitem!\\
enumerate & \verb!\benum...\eenum!\\
description & \verb!\bdesc...\edesc!\\ \midrule
algorithm & \verb!\balg...\ealg!\\
algorithmic & \verb!\balgc...\ealgc!\\ \midrule
table & \verb!\btab...\etab!\\
subtable & \verb!\bsubtab...\esubtab!\\
tabular & \verb!\btabr...\etabr!\\ \midrule
figure & \verb!\bfig...\efig!\\
figure* & \verb!\bfigs...\efigs!\\
subfigure & \verb!\bsubfig...\esubfig!\\ \midrule
center & \verb!\bcent...\ecent!\\ \midrule
align & \verb!\[...\]!\\ 
inline math & \verb!$...$!\\ \midrule
\multicolumn{2}{c}{\emph{Note: These are numbered theorem-like environments.}}\\
\multicolumn{2}{c}{\emph{For unnumbered, append a \texttt{u}: e.g.,} \texttt{bthmu...ethmu}.}\\
theorem & \verb!\bthm...\ethm!\\ 
lemma & \verb!\blem...\elem!\\
proposition & \verb!\bprop...\eprop!\\
corollary & \verb!\bcor...\ecor!\\
conjecture & \verb!\bconj...\econj!\\
definition & \verb!\bdef...\edef!\\
assumption & \verb!\bassump...\eassump!\\
example & \verb!\bexa...\eexa!\\
remark & \verb!\brmk...\ermk!\\
fact & \verb!\bfact...\efact!\\
exercise & \verb!\bexer...\eexer!\\ \midrule
proof & \verb!\bprf...\eprf!\\
proofof & \verb!\bprfof{\cref{theorem_label}}...\eprfof!\\  \midrule
matrix & \verb!\bmat...\emat!\\
bmatrix & \verb!\bbmat...\ebmat!\\
pmatrix & \verb!\bpmat...\epmat!\\
\bottomrule
\etabr
\ecent
}

\newpage
\section{Delimiters}

Mathematics in \LaTeX~often includes quite a few delimiters (parentheses, brackets, curly brackets, etc.).
A very common usage of these involves the \verb!\left...\right! commands for automatic sizing. 
One can also use \verb!\bigl...\bigr!, \verb!\Bigl...\Bigr!, \verb!\biggl...\biggr!, \verb!\Biggl...\Biggr! to control sizing manually.
ShorTeX creates shorthands for these.

\bcent
\btabr{@{}llll@{}}
\toprule
Description & Syntax  \\ \midrule
automatic	& \verb!\lt...\rt!\\        
big 	& \verb!\lb...\rb!\\
Big & \verb!\lB...\rB! \\ 
bigg & \verb!\lbg...\rbg!\\ 
Bigg & \verb!\lBg...\rBg!\\
\bottomrule
\etabr
\ecent

These can be applied to all the usual delimiter characters.
The following tables demonstrate usage for automatically sized delimiters. 

\bcent
\btabr{@{}llll@{}}
\toprule
Description & Example & Text style & Display style \\ \midrule
parentheses	& \verb!\lt(\frac{x}{y}\rt)!        	& $\lt(\frac{x}{y}\rt)$ 		& $\displaystyle\lt(\frac{x}{y}\rt)$ \\[10pt]
curly brackets 	& \verb!\lt\{\frac{x}{y}\rt\}!    	& $\lt\{\frac{x}{y}\rt\}$ 	& $\displaystyle\lt\{\frac{x}{y}\rt\}$ \\[10pt]
square brackets & \verb!\lt[frac{x}{y}\rt]!        	& $\lt[\frac{x}{y}\rt]$ 	& $\displaystyle\lt[\frac{x}{y}\rt]$ \\[10pt]
pipes & \verb!\lt|frac{x}{y}\rt|!        	& $\lt|\frac{x}{y}\rt|$ 	& $\displaystyle\lt|\frac{x}{y}\rt|$ \\[10pt]
double pipes & \verb!\lt\|frac{x}{y}\rt\|!        	& $\lt\|\frac{x}{y}\rt\|$ 	& $\displaystyle\lt\|\frac{x}{y}\rt\|$ \\[10pt]
angle brackets & \verb!\lt<frac{x}{y}\rt>!        	& $\lt<\frac{x}{y}\rt>$ 	& $\displaystyle\lt<\frac{x}{y}\rt>$ \\[10pt]
\bottomrule
\etabr
\ecent

\newpage
\section{Greek characters and variants}\label{sec:greeks}

ShorTeX defines shorthands for Greek characters and variants.
The syntax is just the first two characters of the greek name
(except for omicron, which is identical to the Roman o and O
and so no special characters are required).
Variants are obtained by preceding the usual command with \verb!\v...!.

\bcent
\btabr{@{}lllll@{}}
\toprule
Letter & Syntax & Symbol & Variant Syntax & Variant Symbol  \\ \midrule
alpha & \verb!\al,A! & $\al,A$ &  &  \\
beta & \verb!\be,B! & $\be,B$ &  &  \\
gamma & \verb!\ga,\Ga! & $\ga,\Ga$ &  &  \\
delta & \verb!\de,\De! & $\de,\De$ &  &  \\
epsilon & \verb!\ep,E! & $\ep,E$ & \verb!\vep,E! & $\vep$ \\
zeta & \verb!\ze,Z! & $\ze,Z$ &  &  \\
eta & \verb!\et,H! & $\et,H$ &  &  \\
theta & \verb!\th,\Th! & $\th,\Th$ & \verb!\vth! & $\vth$ \\
iota & \verb!\io,I! & $\io,I$ &  &  \\
kappa & \verb!\ka,K! & $\ka,K$ & \verb!\vka! & $\vka$ \\
lambda & \verb!\la,\La! & $\la,\La$ &  &  \\
mu & \verb!\mu,M! & $\mu,M$ &  &  \\
nu & \verb!\nu,N! & $\nu,N$ &  &  \\
xi & \verb!\xi,\Xi! & $\xi,\Xi$ &  &  \\
omicron & \verb!o,O! & $o,O$ &  &  \\
pi & \verb!\pi,\Pi! & $\pi,\Pi$ & \verb!\vpi! & $\vpi$ \\
rho & \verb!\rh,P! & $\rh,P$ & \verb!\vrh! & $\vrh$ \\
sigma & \verb!\si,\Si! & $\si,\Si$ & \verb!\vsi! & $\vsi$ \\
tau & \verb!\ta,T! & $\ta,T$ &  &  \\
upsilon & \verb!\up,\Up! & $\up,\Up$ &  &  \\
phi & \verb!\ph,\Ph! & $\ph,\Ph$ & \verb!\vph! & $\vph$ \\
chi & \verb!\ch,X! & $\ch,X$ &  &  \\
psi & \verb!\ps,\Psi! & $\ps,\Ps$ &  &  \\
omega & \verb!\om,\Om! & $\om,\Om$ &  &  \\
\bottomrule
\etabr
\ecent

\newpage
\section{Font styles and accents}\label{sec:fontstylesaccents}

Applying accents (e.g., hats $\s[h]a$, tildes $\s[t]a$, bars $\s[b]a$)
and changing fonts (e.g., doublestroke $\s[d]A$, caligraphic $\s[c]A$, and bold $\s[k]A$)
is quite cumbersome in standard \LaTeX. For example, the code to make a tilde caligraphic A,
$\widetilde{\mathcal{A}}$
is \verb!\widetilde{\mathcal{A}}!. By itself that code is not too bad, but many such characters 
in a large mathematical expression results in unreadable code.

ShorTeX defines an efficient syntax for changing fonts and applying accents to characters. 
The syntax takes the form \verb!\s[modifiers]character!, where \verb!modifiers! is a set of single characters
that represent font/accent modifications to \verb!character!. 
For example, the code for tilde caligraphic A is \verb!\s[tc]A! where \verb!t! represents ``tilde,'' \verb!c! represents
``caligraphic,'' and \verb!A! is the character to typeset.

\emph{Note: modifiers are applied in the reverse of the order in which they appear; 
the modifier furthest to the right is applied first. This matches the order that 
the corresponding commands would appear in TeX code.}

\bcent
\btabr{@{}llll@{}}
\toprule
Style/Accent & Modifier & Example & Typeset Example \\ \midrule
caligraphic (\verb!mathcal!) & \verb!c! & \verb!\s[c]A! & $\s[c]A$ \\
bold (\verb!mathbf!) & \verb!k! & \verb!\s[k]A! & $\s[k]A$\\
doublestroke (\verb!mathbb!) & \verb!d! & \verb!\s[d]A! & $\s[d]A$\\
fraktur (\verb!mathfrak!) & \verb!f! & \verb!\s[f]A! & $\s[f]A$\\
hat (\verb!widehat!) & \verb!h! & \verb!\s[h]A! & $\s[h]A$\\
tilde (\verb!widetilde!) & \verb!t! & \verb!\s[t]A! & $\s[t]A$\\
bar (\verb!widebar!) & \verb!b! & \verb!\s[b]A! & $\s[b]A$\\
\bottomrule
\etabr
\ecent

These style modifiers can be combined; the underlying code is flexible enough that
it will happily produce a wide variety of combinations, including those that aren't very sensible.

\bcent
\btabr{@{}llll@{}}
\toprule
Style/Accent & Modifier & Example & Typeset Example \\ \midrule
caligraphic tilde & \verb!ct! & \verb!\s[ct]A! & $\s[ct]A$ \\
bold hat & \verb!kh! & \verb!\s[kh]A! & $\s[kh]A$\\
tilde hat  & \verb!ht! & \verb!\s[ht]A! & $\s[ht]A$\\
hat tilde   & \verb!th! & \verb!\s[th]A! & $\s[th]A$\\
\bottomrule
\etabr
\ecent

We can avoid typing \texttt{[]} for commonly used patterns
by parsing the font style string in advance.
For example, if we use ``bold hat'' symbols frequently,
we might want to use commands like
\verb!\skh...!  instead of \verb!\s[kh]...!.
We can accomplish this using the \verb!\parsefontstylestrings! command,
with syntax
\begin{verbatim}
\parsefontstylestrings{{<fstr1>}{<fstr2}...}{<alphabet>}
\end{verbatim}
For example, to define ``bold hat'' and ``caligraphic hat'' styles
for the characters A, B, C, and D, we would use the command 
\begin{verbatim}
\parsefontstylestrings{{kh}{ch}}{ABCD}
\end{verbatim}
\parsefontstylestrings{{kh}{ch}}{ABCD}
and then in the \LaTeX~document, use the commands
\verb!\skhA \skhB \skhC \skhD! and
\verb!\schA \schB \schC \schD! 
to obtain the following symbols:
\[
\skhA \skhB \skhC \skhD 
\schA \schB \schC \schD 
\]
As another example, for ``bold hat'' applied to $\alpha$, $\beta$, and $\gamma$, we would use the syntax
\begin{verbatim}
\parsefontstylestrings{{kh}}{{\al}{\be}{\ga}}
\end{verbatim}
\parsefontstylestrings{{kh}}{{\al}{\be}{\ga}}
and then in the \LaTeX~document, use the commands
\verb!\skhal \skhbe \skhga!
to obtain the following symbols:
\[
	\skhal \skhbe \skhga
\]

For convenience we also provide a few common alphabets of symbols 
for use in the \verb!\parsefontstylestrings! command.
Note that not every Greek character has a lowercase or uppercase version (in cases where it is
identical to its Roman counterpart). Also note that we use ShorTeX Greek letter syntax;
see \cref{sec:greeks}.

\bcent
\btabr{@{}lll@{}}
\toprule
Syntax & Characters  \\ \midrule
\verb!\lowercaseRoman! & abcdefghijklmnopqrstuvwxyz \\
\verb!\uppercaseRoman! & ABCDEFGHIJKLMNOPQRSTUVWXYZ \\
\verb!\lowercaseGreek! & al,be,ga,de,ep,ze,et,th,io,ka,la,mu,nu,xi,pi,rh\\
& si,ta,up,ph,ch,ps,om,vep,vth,vka,vpi,vrh,vsi,vph\\
\verb!\uppercaseGreek! & Ga,De,Th,La,Xi,Pi,Si,Up,Ph,Ps,Om\\
\bottomrule
\etabr
\ecent

Finally, by default, ShorTeX comes by default with all upper/lowercase Greek and Roman
characters with bold, caligraphic, doublestroke, hat, tilde, bar, and pairwise combinations 
of (bold, caligraphic, doublestroke) with (hat, tilde, bar).   For example,
\verb!\skA!
\verb!\scA!
\verb!\sdA!
\verb!\shA!
\verb!\stA!
\verb!\sbA!
\[
	\skA \scA \sdA \shA \stA \sbA
\]
To disable these default shortcuts, pass the \texttt{minimal} option to ShorTeX.

\newpage
\section{Commenting}\label{sec:commenting}
ShorTeX defines four types of document markup that can be used: 
\emph{comment}, \emph{emphasized comment}, \emph{margin comment}, and \emph{highlight}.
This is a lightweight alternative to some more common todo packages (e.g., \texttt{todonotes}).
In order to create comments, you must pass the \verb!commenters! option to the package, and specify
an identifier for each commenter. For example, one could specify three commenters 
(named A, B, C) using the command\\
\\
\verb!\usepackage[commenters={A,B,C}]{shortex}!
\\\\
For each commenter, there are four commands (one for each markup type). The table
below contains examples for commenter ``A''. Notice that each comment is tagged with a number
(specific to each commenter) for easy referencing.

\bcent
\btabr{@{}llll@{}}
\toprule
Comment Type & Syntax & Example & Typeset Example\\ \midrule
comment & \verb!\c...{comment}! & \verb~\cA{hello!}~ & \cA{hello!} \\ 
emphasized comment & \verb!\e...{comment}! & \verb~\eA{hello!}~ & \eA{hello!} \\ 
margin comment & \verb!\m...{comment}! & text\verb~\mA{hello!}~ & text\mA{hello!} \\ 
highlight & \verb!\h...{text}! & \verb~\hA{hello!}~ & \hA{hello!} \\ 
\bottomrule
\etabr
\ecent

Note that each commenter gets their own color. Currently ShorTeX supports nine different commenter colors,
and will wrap around back to the first color if the number of commenters exceeds nine:
\bitem
\setlength{\itemsep}{0pt}
\item \cA{example}\eA{emphasized example}more text\mA{margin example}
\item \cB{example}\eB{emphasized example}more text\mB{margin example}
\item \cC{example}\eC{emphasized example}more text\mC{margin example}
\item \cD{example}\eD{emphasized example}more text\mD{margin example}
\item \cE{example}\eE{emphasized example}more text\mE{margin example}
\item \cF{example}\eF{emphasized example}more text\mF{margin example}
\item \cG{example}\eG{emphasized example}more text\mG{margin example}
\item \cH{example}\eH{emphasized example}more text\mH{margin example}
\item \cI{example}\eI{emphasized example}more text\mI{margin example}
\item Back to first color: \cJ{example}\eJ{emphasized example}more text\mJ{margin example}
\eitem
You can also disable all comments to see a clean version of the current 
document using the \verb!suppresscomments! package option.
This option will blank out all comments and render highlighted text normally.

\newpage
\section{Referencing figures, equations, tables, etc.}

ShorTeX includes the \texttt{cleveref} package, which simplifies the process
of typesetting references. Use the \verb!\cref! command (or \verb!\Cref! at the beginning of a sentence) 
to automatically typeset the names of the objects you reference (including properly handling multiple references). 
For example, if \verb!\label{fig:first}! is applied to the first figure in the document,
\begin{verbatim}
In \cref{fig:first}, you can see...
\end{verbatim}
would typeset as ``In Fig.~1, you can see...''
Similarly, if \verb!\label{thm:first}! references a theorem and \verb!\label{second_result}! references
a lemma, 
\begin{verbatim}
\cref{thm:first,lem:second} show that...
\end{verbatim}
will typeset as ``Theorem 1 and Lemma 2 show that...''
This works for many different reference types (Figure, Algorithm, Equation, Table, etc),
and can be extended if needed. See the \texttt{cleveref} documentation 
at \url{https://ctan.org/pkg/cleveref?lang=en} and the homepage at \url{https://www.dr-qubit.org/cleveref.html} 
for more information.

ShorTeX also includes the \texttt{autonum} package, which simplifies the process of 
equation numbering. Typically when you typeset equations, you have to choose between 
\verb!$...$!, \verb!$$...$$!, 
\verb!\begin{align}...\end{align}!, 
\verb!\begin{aligned}...\end{aligned}!, 
\verb!\begin{equation}...\end{equation}!, 
not to mention starred versions of those environments 
and \verb!\nonumber!/\verb!\notag! commands, depending 
on whether/where you want equation numbers,

The \texttt{autonum} package automatically decides which equations to provide
numbers based on \textit{which equations you reference}. So when using ShorTeX,
you only need two commands for math mode: single dollar signs \verb!$...$! for
inline math, and \texttt{align} environments (redefined in ShorTeX to be
\verb!\[...\]!) for display math.\footnote{There are
differences between how \texttt{align} and 
other math display environments typeset equations. I have not ever
encountered a case where it mattered much. If you are very picky about typesetting,
note that ShorTeX does not \emph{disable} any functionality, so you 
can use other environments anywhere you feel it is necessary.}

For example, if you create 
the following display math,
\begin{verbatim}
\[
   a+b = c \label{eq:the_equation}
\]
\end{verbatim}
then if you use the command \verb!\cref{eq:the_equation}! somewhere
in the document, that equation will automatically be assigned a number. If not, it
won't get a number. See the \texttt{autonum} package 
documentation \url{https://ctan.org/pkg/autonum?lang=en} for more information.



\section{Custom macros}

\subsection{Shrinking whitespace in math}
The command \verb!\squish{<frac>}! in math mode enables you to shrink whitespace in mathematics,
where \verb!<frac>! represents the fraction of whitespace reduction.
Below, the first line is regularly spaced, the second line has \verb!\squish{0.5}!, and the third has \verb!\squish{0.0}!.
\[
	\sqrt{\frac{1^{2}}{0.111222}(0.111222\times1.111163+0.066987^{2}\times0.111222)-1}&= \sqrt{0.111222}\\
	\squish{0.5}\sqrt{\frac{1^{2}}{0.111222}(0.111222\times1.111163+0.066987^{2}\times0.111222)-1}&= \sqrt{0.111222}\\
	\squish{0.0}\sqrt{\frac{1^{2}}{0.111222}(0.111222\times1.111163+0.066987^{2}\times0.111222)-1}&= \sqrt{0.111222}\\
\]

The code for \verb!\squish! was taken from \url{https://tex.stackexchange.com/questions/467942/how-to-squeeze-a-long-equation}.

\subsection{Wide bar}

ShorTeX provides the \verb!\widebar! command to typeset a wide bar accent on top of a character (similar to the 
usual \verb!\widehat! and \verb!\widetilde! commands). Compare to the usual \verb!\bar! and 
\verb!\overline! commands:
\[
	\text{\texttt{widebar}:}\,\, \widebar{A} \qquad \text{\texttt{overline}:} \,\,\overline{A} \qquad \text{\texttt{bar}:} \,\,\bar{A}
\]
The code for \verb!\widebar! was taken from \url{https://tex.stackexchange.com/questions/16337/can-i-get-a-widebar-without-using-the-mathabx-package}.
Note that the shortened style/accent code \texttt{b} in \cref{sec:fontstylesaccents} encodes \verb!\widebar!, not \verb!\bar!.


\subsection{Sets and set operations}
\bcent
\btabr{@{}lll@{}}
\toprule
Name & Syntax & Symbol  \\ \midrule
reals	& \verb!\reals! & $\reals$ \\
extended reals	& \verb!\extreals! & $\extreals$ \\
rationals & \verb!\rats! & $\rats$\\
integers	& \verb!\ints! & $\ints$ \\
natural numbers	& \verb!\nats! & $\nats$ \\
complex numbers	& \verb!\comps! & $\comps$ \\
measures & \verb!\measures! & $\measures$\\
probability measures & \verb!\pmeasures! & $\pmeasures$\\
(big) intersection & \verb!\intersect!, \verb!\Intersect! & $\intersect$, $\Intersect$\\
(big) union & \verb!\union!, \verb!\Union! & $\union$, $\Union$\\
(big) disjoint union & \verb!\djunion!, \verb!\djUnion! & $\djunion$, $\djUnion$\\
volume	& \verb!\vol! & $\vol$ \\
diameter	& \verb!\diam! & $\diam$ \\
boundary	& \verb!\boundary! & $\boundary$ \\
closure	& \verb!\closure! & $\closure$ \\
span	& \verb!\spann! & $\spann$ \\
cone	& \verb!\cone! & $\cone$ \\
convex hull	& \verb!\conv! & $\conv$ \\
\bottomrule
\etabr
\ecent

\subsection{Linear algebra}

\bcent
\btabr{@{}lll@{}}
\toprule
Name & Syntax & Symbol  \\ \midrule
trace	& \verb!\tr A! & $\tr A$ \\
rank	& \verb!\rank A! & $\rank A$ \\
transpose	& \verb!A\T! & $A\T$ \\
inverse transpose	& \verb!A\nT! & $A\nT$ \\
diagonal	& \verb!\diag A! & $\diag A$ \\
adjoint	& \verb!A\adj! & $A\adj$ \\
spectrum	& \verb!\spec A! & $\spec A$ \\
kronecker product & \verb!A\kron B! & $A\kron B$\\
\bottomrule
\etabr
\ecent

\subsection{Calculus}

\bcent
\renewcommand{\arraystretch}{1.5}
\btabr{@{}lll@{}}
\toprule
Name & Syntax & Symbol  \\ \midrule
differential symbol	& \verb!\dee x! & $\dee x$ \\
gradient symbol	& \verb!\grad f! & $\grad f$ \\
derivative	& \verb!\der{x}{y}! & $\der{x}{y}$ \\
double derivative	& \verb!\dder{x}{y}! & $\dder{x}{y}$ \\
derivative w.r.t.	& \verb!\derwrt{y}! & $\derwrt{y}$ \\
partial derivative	& \verb!\pder{x}{y}! & $\pder{x}{y}$ \\
partial double derivative	& \verb!\pdder{x}{y}! & $\pdder{x}{y}$ \\
$i^\text{th}$ partial derivative & \verb!\pderi{x}{y}{i}! & $\pderi{x}{y}{i}$ \\
partial derivative w.r.t. & \verb!\pderwrt{y}! & $\pderwrt{y}$ \\
Hessian & \verb!\hes{a}{x}{y}! & $\hes{a}{x}{y}$ \\
\bottomrule
\etabr
\ecent

\subsection{General mathematics}

\bcent
\btabr{@{}lll@{}}
\toprule
Name & Syntax & Symbol  \\ \midrule
argmax	& \verb!\argmax_{x\in \reals}! & $\argmax_{x\in\reals}$ \\
argmin	& \verb!\argmin_{x\in \reals}! & $\argmin_{x\in\reals}$ \\
esssup	& \verb!\esssup_{x\in \reals}! & $\esssup_{x\in\reals}$ \\
essinf	& \verb!\essinf_{x\in \reals}! & $\essinf_{x\in\reals}$ \\
indicator	& \verb!\ind[x=3]! & $\ind[x=3]$ \\
sign	& \verb!\sgn x! & $\sgn x$ \\
scientific notation	& \verb!\scin{3}{5}! & $\scin{3}{5}$ \\
%such that	& \verb!\st! & $\st$ \\
given	& \verb!\given! & $\given$ \\
defined as	& \verb!\defas! & $\defas$ \\
defines	& \verb!\defines! & $\defines$ \\
half	& \verb!\half! & $\half$ \\
third	& \verb!\third! & $\third$ \\
quarter	& \verb!\quarter! & $\quarter$ \\
\bottomrule
\etabr
\ecent

\subsection{Common words and names with accents}

\bcent
\btabr{@{}ll@{}}
\toprule
 Syntax & Symbol  \\ \midrule
\verb!\cadlag! & \cadlag \\
\verb!\Frechet! & \Frechet \\
\verb!\Gronwall! & \Gronwall \\
\verb!\Holder! & \Holder \\
\verb!\Ito! & \Ito \\
\verb!\Levy! & \Levy \\
\verb!\Matern! & \Matern \\
\verb!\Nystrom! & \Nystrom \\
\verb!\Renyi! & \Renyi \\
\verb!\Schatten! & \Schatten \\
\bottomrule
\etabr
\ecent


\subsection{Probability and statistics}



\bcent
\btabr{@{}lll@{}}
\toprule
Name & Syntax & Symbol  \\ \midrule
i.i.d.	& \verb!\iid! & \iid \\
almost sure	& \verb!\as! & \as \\
almost everywhere	& \verb!\aev! & \aev \\
convergence almost surely	& \verb!\convas! & $\convas$ \\
convergence in probability	& \verb!\convp! & $\convp$ \\
convergence in distribution	& \verb!\convd! & $\convd$ \\
equality in distribution	& \verb!\eqd! & $\eqd$ \\
equality almost surely	& \verb!\eqas! & $\eqas$ \\
probability	& \verb!\P! & $\P$ \\
expectation	& \verb!\E! & $\E$ \\
variance	& \verb!\var! & $\var$ \\
covariance	& \verb!\cov! & $\cov$ \\
correlation	& \verb!\cor! & $\cor$ \\
support	& \verb!\supp! & $\supp$ \\
distributed as	& \verb!\dist! & $\dist$ \\
distributed \iid	& \verb!\distiid! & $\distiid$ \\
distributed independently	& \verb!\distind! & $\distind$ \\
independent &  \verb!\indep! & $\indep$ \\
Entropy & \verb!\ent{q}! & $\ent{q}$\\
KL divergence & \verb!\kl{q}{p}!, \verb!\kl[a]{q}{p}! & $\kl{q}{p}$, $\kl[a]{q}{p}$\\
Hellinger distance & \verb!\hell{q}{p}!, \verb!\hell[a]{q}{p}! & $\hell{q}{p}$, $\hell[a]{q}{p}$\\
Total variation distance & \verb!\tvd{q}{p}!, \verb!\tvd[a]{q}{p}! & $\tvd{q}{p}$, $\tvd[a]{q}{p}$\\
\bottomrule
\etabr
\ecent


\bcent
\btabr{@{}lll@{}}
\toprule
Name & Syntax & Symbol  \\ \midrule
Bernoulli	& \verb!\distBern! & $\distBern$ \\
beta	& \verb!\distBeta! & $\distBeta$ \\
beta prime	& \verb!\distBetaPrime! & $\distBetaPrime$ \\
binomial	& \verb!\distBinom! & $\distBinom$ \\
categorical	& \verb!\distCat! & $\distCat$ \\
Cauchy	& \verb!\distCauchy! & $\distCauchy$ \\
chi-squared	& \verb!\distChiSq! & $\distChiSq$ \\
Dirichlet	& \verb!\distDir! & $\distDir$ \\
exponential	& \verb!\distExp! & $\distExp$ \\
gamma	& \verb!\distGam! & $\distGam$ \\
inverse gamma	& \verb!\distInvGam! & $\distInvGam$ \\
geometric	& \verb!\distGeom! & $\distGeom$ \\
Gumbel	& \verb!\distGum! & $\distGum$ \\
generalized extreme value	& \verb!\distGEV! & $\distGEV$ \\
Laplace	& \verb!\distLap! & $\distLap$ \\
multinomial	& \verb!\distMulti! & $\distMulti$ \\
normal	& \verb!\distNorm! & $\distNorm$ \\
Poisson	& \verb!\distPoiss! & $\distPoiss$ \\
student-t	& \verb!\distT! & $\distT$ \\
uniform	& \verb!\distUnif! & $\distUnif$ \\
von Mises-Fisher	& \verb!\distVMF! & $\distVMF$ \\
Wishart	& \verb!\distWish! & $\distWish$ \\
inverse Wishart	& \verb!\distInvWish! & $\distInvWish$ \\
\midrule
Bernoulli process	& \verb!\distBeP! & $\distBeP$ \\
beta process	& \verb!\distBP! & $\distBP$ \\
beta prime process	& \verb!\distBPP! & $\distBPP$ \\
Dirichlet process	& \verb!\distDP! & $\distDP$ \\
Chinese restauarant process	& \verb!\distCRP! & $\distCRP$ \\
completely random measure	& \verb!\distCRM! & $\distCRM$ \\
normalized completely random measure & \verb!\distNCRM! & $\distNCRM$ \\
gamma process	& \verb!\distGamP! & $\distGamP$ \\
normalized gamma process	& \verb!\distNGamP! & $\distNGamP$ \\
Gaussian process	& \verb!\distGP! & $\distGP$ \\
Pitman-Yor process	& \verb!\distPYP! & $\distPYP$ \\
Poisson process	& \verb!\distPP! & $\distPP$ \\
\bottomrule
\etabr
\ecent




\subsection{Vector spaces and operators}

\begin{center}
\begin{tabular}{@{}lll@{}}
\toprule
Description						& Syntax 				& Symbol \\ \midrule
Norm					& \verb!\norm{\frac{x}{y}}!       	& $\norm{\frac{x}{y}}$ 	\\[10pt]
Norm  with subscript				& \verb!\normsub*{\frac{x}{y}}{2}!       	& $\normsub*{\frac{x}{y}}{2}$ 	\\[10pt]
Inner product			& \verb!\inner{\frac{x}{y}}{\frac{y}{z}}!       	& $\inner{\frac{x}{y}}{\frac{y}{z}}$ 	\\[10pt]
Inner prod with subscript 		& \verb!\innersub*{\frac{x}{y}}{z}{2}!       	& $\innersub*{\frac{x}{y}}{z}{2}$  \\[10pt]
$\Lp{p}$ space					& \verb!\Lp{2}!        		& $\Lp{2}$ 			\\[10pt]
\begin{tabular}[c]{@{}l@{}}$\Lp{p}$ space for \\ specified measure	 \end{tabular}	& \verb!\Lpm{2}{\mu}!	& $\Lpm{2}{\mu}$ 		\\[10pt]
							& \verb!\Lpm*{2}{\mu}!	& $\Lpm*{2}{\mu}$ 	\\[10pt]
							& \verb!\Lpm[\Big]{2}{\mu}!	& $\Lpm[\Big]{2}{\mu}$ 	\\[10pt]
$\Lp{p}$ norm					& \verb!\Lpnorm{\Gamma}{2}!        & $\Lpnorm{\Gamma}{2}$ 		 \\[10pt]
						& \verb!\Lpnorm*{\Gamma}{2}!        & $\Lpnorm*{\Gamma}{2}$ \\[10pt]
						& \verb!\Lpnorm[\Big]{\Gamma}{2}!        & $\Lpnorm[\Big]{\Gamma}{2}$ \\[10pt]
\begin{tabular}[c]{@{}l@{}}$\Lp{p}$ norm for \\ specified measure	 \end{tabular}		& \verb!\Lpmnorm{\Gamma}{2}{\mu}!        & $\Lpmnorm{\Gamma}{2}{\mu}$  \\[10pt]
							& \verb!\Lpmnorm*{\Gamma}{2}{\mu}!        & $\Lpmnorm*{\Gamma}{2}{\mu}$ \\[10pt]
							& \verb!\Lpmnorm[\Big]{\Gamma}{2}{\mu}!        & $\Lpmnorm[\Big]{\Gamma}{2}{\mu}$ \\[10pt]
$\Lp{p}$ inner product				& \verb!\Lpinner{\Gamma}{\Gamma}{2}!        & $\Lpinner{\Gamma}{\Gamma}{2}$ 	\\[10pt]
							& \verb!\Lpinner*{\Gamma}{\Gamma}{2}!        & $\Lpinner*{\Gamma}{\Gamma}{2}$  \\[10pt]
							& \verb!\Lpinner[\Big]{\Gamma}{\Gamma}{2}!        & $\Lpinner[\Big]{\Gamma}{\Gamma}{2}$  \\[10pt]
\begin{tabular}[c]{@{}l@{}}$\Lp{p}$ inner product \\  for specified measure	 \end{tabular} & \verb!\Lpminner{\Gamma}{\Gamma}{2}{\mu}!        & $\Lpminner{\Gamma}{\Gamma}{2}{\mu}$ 	 \\[10pt]
							& \verb!\Lpminner*{\Gamma}{\Gamma}{2}{\mu}!        & $\Lpminner*{\Gamma}{\Gamma}{2}{\mu}$ 		 \\[10pt]
							& \verb!\Lpminner[\big]{\Gamma}{\Gamma}{2}{\mu}!        & $\Lpminner[\big]{\Gamma}{\Gamma}{2}{\mu}$ 		 \\[10pt]
\bottomrule
\end{tabular}
\end{center}


\subsection{Paired delimiters}

\begin{center}
\begin{tabular}{@{}llll@{}}
\toprule
Description				& Example 					& Text style 				& Display style \\ \midrule
Round brackets	& \verb!\rbra{\frac{x}{y}}!        	& $\rbra{\frac{x}{y}}$ 		& $\displaystyle\rbra{\frac{x}{y}}$ \\[10pt]
Curly brackets 			& \verb!\cbra*{\frac{x}{y}}!    	& $\cbra*{\frac{x}{y}}$ 	& $\displaystyle\cbra*{\frac{x}{y}}$ \\[10pt]
Square brackets 			& \verb!\sbra[\bigg]{\frac{x}{y}}!        	& $\sbra[\bigg]{\frac{x}{y}}$ 	& $\displaystyle\sbra[\bigg]{\frac{x}{y}}$ \\[10pt]
Absolute value 			& \verb!\abs{\frac{x}{y}}!        	& $\abs{\frac{x}{y}}$ 		& $\displaystyle\abs{\frac{x}{y}}$ \\[10pt]
Set 					& \verb!\set{\frac{x}{y}, \frac{y}{z}}!        & $\set{\frac{x}{y}, \frac{y}{z}}$ 	& $\displaystyle\set{\frac{x}{y}, \frac{y}{z}}$ \\[10pt]
Floor					& \verb!\floor{\frac{x}{y}}!        	& $\floor{\frac{x}{y}}$ 		& $\displaystyle\floor{\frac{x}{y}}$ \\[10pt]
Ceiling 				& \verb!\ceil{\frac{x}{y}}!        	& $\ceil{\frac{x}{y}}$ 		& $\displaystyle\ceil{\frac{x}{y}}$ \\[10pt]
Cardinality 				& \verb!\card{\s[h]A}!       		& $\card{\s[h]A}$ 			& $\displaystyle\card{\s[h]A}$ \\[10pt]
\bottomrule
\end{tabular}
\end{center}


\section{Example Document}

TODO: full shortex example

\end{document}

